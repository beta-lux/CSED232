\documentclass{article}
\usepackage[T1]{fontenc}
\usepackage[utf8]{inputenc}
\usepackage{amsmath}
\usepackage{amsfonts}
\usepackage{fontspec}
\usepackage{amsthm}
\usepackage{amssymb}
\usepackage{stackengine}
\usepackage{minted}
\usepackage{mathtools}
\usepackage{graphicx,wrapfig,tikz}
\usepackage{physics}
\setlength\intextsep{0pt}
\usetikzlibrary{calc,hobby}
\usepackage{pgfplots}
\usepackage{xcolor}
\usepackage{xargs}
\usepackage{kotex}
\usepackage{mathrsfs} %fourier transform
\usetikzlibrary{external}
\usetikzlibrary{arrows}
\usepackage{setspace}
\usetikzlibrary{arrows}
\usepackage{tabstackengine}
\usepackage{enumitem}
\usepackage{longtable}
\usepackage{listings}
\setlist[enumerate]{itemsep=0mm}
\setlist[enumerate]{label={\arabic*.}}

\definecolor{beta-gray}{rgb}{0.9,0.9,0.9}

\newcommand{\lt}{\;\,}
\newcommand{\ft}{\quad\!}
\newcommand{\ot}{\quad\;}
\newcommand{\dt}{\;\,\;\,}
\newcommand{\plm}[1]{\setcounter{enumi}{#1 - 1}}
\newcommand{\nd}{\quad \text{and} \quad}
\newcommand{\inline}[1]{%
    \colorbox{beta-gray}{\lstinline[language=C++]{#1}}%
}

\newenvironment{code}[5]{
    {
        \usemintedstyle{xcode}
        \fontspec{Menlo}
        \begin{spacing}{1.0}
            \let\itshape\relax
            \inputminted[frame=single, firstline=#3, lastline=#4]{#1}{#2}
        \end{spacing}
        \vspace{#5}
    }

}

\newenvironment{answer}{
    {
    \newline
        \textbf{Answer.}\\
    }
    {}
}

\newenvironment{beka}{

}

\newenvironment{answer*}{
    {
        \newline
        \textbf{Answer.}
    }
    {}
}

\usepackage[margin=1in]{geometry}

\begin{document}
    \author{\large 권민재, 20190084}
    \title{\Large\textbf{CSED232: Object-Oriented Programming Final Exam}}
    \date{\small June 27, 2020}
    \maketitle

    \setstretch{1.5}

    \begin{enumerate}[itemsep=30pt]
        \item 다음 개념들이 각각 무엇을 뜻하는지 한두줄 내외로 간략히 기술하시오.(각 4점, 총 28점)
        \begin{enumerate}[label={\Alph*.}, itemsep=15pt]
            \item Pure virtual function
            \begin{answer}
                Pure virtual function은, 이것이 선언된 클래스에서 함수의 내용이 구현되지 않고, derived 클래스에서 재정의해서 내용을 구현해야 하는 virtual function이다. C++에서는 `=0'을 붙여서 pure virtual function임을 명시할 수 있다.
            \end{answer}
            \item Function overloading
            \begin{answer}
                Function overloading은 함수의 이름은 같지만, signature가 다른 여러 함수를 만들 수 있는 기능을 일컫는다.
                overloading된 함수가 불릴 때, 주어진 매개변수에 대해 일종의 함수 리스트들 중에서 적절한 함수가 존재할 때에만 그것이 실행된다.
            \end{answer}
            \item Function overriding
            \begin{answer}
                Function overriding은 derived 클래스에서 base 클래스에 작성/선언된 function(method)을 재정의(redefine)하는 것이다.
            \end{answer}
            \item Dynamic binding
            \begin{answer}
                Dynamic binding은 런타임에 바인딩을 하는 것을 뜻하며, 주로 virtual function을 바인딩할 때 일어난다. Dynamic binding을 위해서 컴파일러는 프로그램이 실행될 때 올바른 virtual method가 선택될 수 있도록 코드를 생성해야한다.
            \end{answer}
            \item Abstract base class
            \begin{answer}
                Abstract base class는 Pure virtual functions로 구성된 base class로, abstract base class의 오브젝트는 생성할 수 없다. 즉, base class로만 사용되는 클래스로, 다른 derived class를 생성할 때 사용된다.
            \end{answer}
            \item Generic programming
            \begin{answer}
                Generic programming이란, 코드를 type(자료형)에 독립적이도록 작성하는 것이다. 변수들이 하나의 자료형에 국한되지 않고, 여러 자료형을 가질 수 있도록 하여 재사용성을 높일 수 있는 프로그래밍 방식이다.
            \end{answer}
            \item Event-driven programming
            \begin{answer}
                Event-driven programming이란 마우스 클릭,키 입력과 같은 ‘event’가 일어났을 경우 그에 상응하는 일을 하도록 코드를 작성하는 프로그래밍 방식이다.
            \end{answer}
        \end{enumerate}
        \item C++ Standard Template Library 의 algorithm 들은 iterator 를 사용하도록 설계되어 있다. iterator 를 사용하는 것의 장점이 무엇인지에 대해 간략하고 명확하게 기술하시오. (4 점)
        \begin{answer}
            STL이 다양한 컨테이너 클래스에서 포인터의 일반화된 형태인 iterator를 통해 uniform한 인터페이스를 제공하는 것에서 볼 수 있듯이, (1) iterator를 통해 포인터를 일반화하는 것 그 자체가 큰 장점이다.
            (1-1) 일반화되어 있기 때문에, vector와 같은 STL 컨테이너를 다른 컨테이너로 변경하는데 노력이 적게 들어간다.
            (1-2) 또한, <algorithm>에 미리 작성된 여러 함수들을 iterator를 이용하여 사용할 수도 있다.
            (1-3) 즉, 일반화된 형태라는 점을 이용하여 재사용성이 높은 코드를 작성하고 iterator를 통해 이를 이용할 수 있다.
            그리고 (2) iterator를 이용하여 vector와 같은 컨테이너를 순회하면, 범위 안에서만 탐색하기 때문에 자연스럽게 out of range 오류를 줄일 수 있는 장점 또한 존재한다.
        \end{answer}
        \item Has-a relationship 을 구현하기 위한 두 가지 대표적인 방법으로 containment 와 private inheritance 가 있다. Has-a relationship 이 무엇인지 기술하고, 또한 containment 와 private inheritance 의 장단점을 비교하시오. (8 점)
        \begin{answer}
            Has-a relationship이란 특정 객체가 다른 객체를 소유하는 관계를 말한다. 예를 들어 Lunch 클래스와 Fruit 클래스가 있다고 해보자. 이때, Lunch로 Fruit를 먹는 것은 적합하지만, 그렇다고 해서 `Lunch is a Fruit'인 것은 아니다. 이들 사이의 관계를 묘사하기에는 `A Lunch has a Fruit'와 같은 문장이 적합하며, 이러한 관계를 has-a 관계로 볼 수 있다.
            이때, containment는 어떤 객체의 멤버로 다른 객체를 포함하여 has-a 관계를 만들고,
            private inheritance는 어떤 클래스가 다른 클래스의 public 멤버들을 private 멤버로 상속함으로써 has-a 관계를 만든다.
            Containment를 통해 has-a 관계를 만들 경우에, contained class의 이름을 명시적으로 지정하고 이용할 수 있기 때문에 더 이해하기 쉽고, 여러 클래스의 상속으로부터 오는 문제가 일어나지 않는다는 장점이 있다. 하지만, containment를 통해서는 다른 클래스의 protected member를 사용할 수 없고, virtual member를 다시 정의할 수 없다는 단점이 있다.
            Private inheritance를 통해 has-a 관계를 만들 경우에, 다른 클래스의 protected member를 사용할 수 있고, 다른 클래스의 virtual function을 다시 정의할 수 있다는 장점이 있다. 하지만, 여러 클래스로부터 상속 받을 경우에 이름이 겹친다든지 하는 여러 문제가 일어날 가능성이 커진다는 단점이 있다.
        \end{answer}
        \item Private inheritance 와 protected inheritance 의 차이점을 자세히 기술하시오. (4 점)
        \begin{answer}
            base class의 protected, public 멤버들이 Private inheritance의 경우 derived class의 private 멤버로 상속되며,
            protected inheritance의 경우 derived class의 protected member로 상속된다.
            즉, protected inheritance인 경우 derived class와 그의 dervied class들이 base class의 public 멤버를 이용할 수 있고,
            private inheritance인 경우 derived class만 base class의 public 멤버를 이용할 수 있다.

        \end{answer}
        \item 다음 코드를 실행시켰을 때 출력되는 내용은 무엇인가? 그리고 왜 그런 출력이 나오는가? 다음 코드를 실행시켰을 때 출력될 내용을 적고, 그 이유를 간단히 기술하시오. (8 점)
              \code{cpp}{05.cpp}{2}{}{15pt}
              \textbf{Answer.}
              \begin{beka}
                  \code{raw}{05.out}{}{}{1pt}
                  pt의 vptr이 동적할당된 DDerived 오브젝트의 vtable을 가리키기 때문에, 위와 같은 결과가 나온다. 자세한 과정을 살펴보면 아래와 같다.\\
                  우선 DDerived 타입의 오브젝트가 동적할당되는 과정을 살펴보자. Derived1에는 3, pt의 Derived2에는 5가 각자의 생성자에 전달된다.
                  이때, Derived1의 생성자에서는 3을 자신의 n에 저장하고, Derived1의 Base의 생성자에 6을 전달한다. 그러면, Derived1의 Base의 n에 6이 저장된다.
                  Derived2의 생성자에서는 5를 자신의 n에 저장하고, Derived2의 Base의 생성자에 15를 전달한다. 그러면, Derived2의 Base의 n에 15가 저장된다.
                  Derived1과 Derived2가 가리키는 Base는 각각 다르다는 것에 유의하자. 이 DDerived 타입으로 동적할당된 오브젝트 주소가 Derived1 자료형의 unique\_ptr, 즉 pt에 저장된다. \\
                  그 후, pt를 통해 display 메서드를 부르는 과정을 살펴보자. pt->display()를 할때, pt의 vptr을 이용해 vtable을 참조하게 되며, 이 vtable은 동적할당된 DDerived 오브젝트의 vtable인 것을 알 수 있다.
                  결과적으로, 이때 vtable에서 display()를 찾아서 실행하는 것은, 위에서 동적 할당한 DDerived 오브젝트의 display()를 호출하는 결과를 만든다. 그래서 pt->display()를 할때, 동적할당된 DDerived 오브젝트의 display()가 실행되며, 이때 Derived1의 display가 호출되고, Derived1의 Base의 display가 호출되며 Base: 6이 출력된다.
                  그 후, Derived1의 display에서 Derived1: 3이 출력된다. 다음으로, Derived2의 display가 호출되고, Derived2의 Base의 display가 호출되며 Base: 15가 출력된다. 그 후, Derived2의 display에서 Derived2: 5가 출력된다. 이 과정을 거쳐서, 위의 결과가 출력되고, 프로그램이 종료된다.
              \end{beka}
        \item 다음은 재귀 호출을 이용하여 factorial 을 계산하는 코드이다. \code{cpp}{06-plm.cpp}{}{}{15pt} 이를 template 을 이용하여 아래와 같이 바꿀 수 있다. 이렇게 template 을 이용하는 방법의 장점은 실제 factorial 계산을 런타임에 하는 대신 컴파일 타임에 할 수 있어서 프로그램의 실행시간을 단축시킬 수 있다는 장점이 있다. 이 때 빈칸에 필요한 코드를 template specialization 을 이용하여 작성하시오. (8 점)
              \begin{answer*}\code{cpp}{06.cpp}{2}{}{0pt}\end{answer*}
        \item 아래의 코드는 C++11 표준 라이브러리에서 제공하는 스마트 포인터 중 하나인 unique\_ptr 을 간략화하여 구현한 클래스이다. 아래의 코드를 완성하기 위한 move constructor 와 move assignment operator 의 코드를 작성하시오. (8 점)
              \begin{answer*}\code{cpp}{07.cpp}{}{}{15pt}\end{answer*}
        \item 다음 코드에는 런타임에 발생할 수 있는 한가지 문제점이 존재한다. 이 문제점이 무엇인가? 그리고 이 문제점을 해결하기 위해서는 코드를 어떻게 수정해야 하는가? SomeClass 와 SomeException 클래스는 모두 적절히 정의되어 있다고 가정한다. (8 점)
              \begin{answer*}
                  \code{cpp}{08-sol.cpp}{32}{45}{5pt}
                  throw를 할 경우에는 가장 가까운 catch로 점프하기 때문에, 문제와 같은 상황에서 b\_error == true 인 경우에, foo()에서 main()의 catch로 점프하는 과정에서 bar()의 p에 동적 할당 된 메모리가 할당이 해제되지 않는 문제가 있다.
                  main()에서 e.what()을 출력하는 등의 에러 처리를 유지하면서, bar()의 p에 할당된 메모리를 할당 해제 해주기 위해서는, 아래 코드와 같이 bar()에서 foo를 실행할 때 try-catch로 감싸고, catch에서 p를 할당 해제 한 후 e를 throw 하는 방식으로 수정하여 해결해야 한다.
              \end{answer*}


        \item 다음 코드에서 C-style 의 type cast 연산자들을 각각 그에 대응하는 static\_cast, const\_cast, reinterpret\_cast로 변경한 코드를 작성하시오. (8 점)
              \begin{answer*}\code{cpp}{09.cpp}{}{}{15pt}\end{answer*}
        \item 다음 코드의 Vec2D 는 2 차원 벡터를 다루기 위한 template class 이다. 아래 코드의 main 함수에서는 Vec2D<int>와 Vec2D<float>의 서로 다른 데이터 타입의 Vec2D 오브젝트 사이의 덧셈 연산을 수행하는 것을 보여준다. 이 덧셈 연산을 구현하기 위한 operator+ 의 오버로딩이 파란색으로 표시되어 있다. 여기서 이 코드에 필요한 적절한 prototype 을 빈칸에 추가하시오. 오버로딩된 operator+는 template function 으로 되어 있으며, Vec2D<int>와 Vec2D<float>, 또는 Vec2D<double>과 Vec2D<long>과 같은 임의의 서로 다른 데이터 타입의 Vec2D 오브젝트 간의 덧셈 연산을 지원한다. 또한 Vec2D<int> 오브젝트와 Vec2D<float> 오브젝트의 덧셈 연산 시 결과로 나와야 되는 타입 (Vec2D<float>) 이 자동으로 결정된다. (4 점)
              \begin{answer*}\code{cpp}{10.cpp}{2}{}{15pt}\end{answer*}
        \item 다음 코드의 지역 변수인 data 는 integer 값들을 담고 있는 vector 이다. 아래 코드에서 파란 색으로 표시된 코드는 이 data 의 각각의 값들의 조사하여 upper\_bound 보다 큰 값들은 upper\_bound 로 값을 변경하고, lower\_bound 보다 작은 값들은 lower\_bound 로 변경하는 작업을 수행한다. 아래 파란색으로 표시된 코드와 동일한 작업을 수행하는 3 가지의 코드를 작성하시오. 각각의 코드는 아래에 A,B,C 로 표시된 내용을 각각 활용하여 작성되어야 한다. (12 점)
              \begin{enumerate}[label={\Alph*.}, itemsep=15pt]
                  \item for-loop 와 iterator 이용           \begin{answer*}\code{cpp}{11-A.cpp}{}{}{15pt}\end{answer*}
                  \item std::transform 함수와 lambda 함수 이용 \begin{answer*}\code{cpp}{11-B.cpp}{}{}{15pt}\end{answer*}
                  \item std::for\_each 함수와 lambda 함수 이용 \begin{answer*}\code{cpp}{11-C.cpp}{}{}{15pt}\end{answer*}
              \end{enumerate}
    \end{enumerate}
\end{document}